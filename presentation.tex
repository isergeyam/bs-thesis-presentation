%%%%%%%%%%%%%%%%%%%%%%%%%%%%%%%%%%%%%%%%%
% Beamer Presentation
% LaTeX Template
% Version 1.0 (10/11/12)
%
% This template has been downloaded from:
% http://www.LaTeXTemplates.com
%
% License:
% CC BY-NC-SA 3.0 (http://creativecommons.org/licenses/by-nc-sa/3.0/)
%
%%%%%%%%%%%%%%%%%%%%%%%%%%%%%%%%%%%%%%%%%

%----------------------------------------------------------------------------------------
%	PACKAGES AND THEMES
%----------------------------------------------------------------------------------------

\documentclass{beamer}

\mode<presentation> {
\usetheme{Boadilla}

%Russian-specific packages
%--------------------------------------
\usepackage[T2A]{fontenc}
\usepackage[utf8]{inputenc}
\usepackage[russian]{babel}
\usepackage{mathrsfs}
\usepackage[backend=biber]{biblatex}
\addbibresource{presentation.bib}
%--------------------------------------

% As well as themes, the Beamer class has a number of color themes
% for any slide theme. Uncomment each of these in turn to see how it
% changes the colors of your current slide theme.

%\usecolortheme{albatross}
%\usecolortheme{beaver}
%\usecolortheme{beetle}
%\usecolortheme{crane}
%\usecolortheme{dolphin}
%\usecolortheme{dove}
%\usecolortheme{fly}
%\usecolortheme{lily}
%\usecolortheme{orchid}
%\usecolortheme{rose}
%\usecolortheme{seagull}
%\usecolortheme{seahorse}
%\usecolortheme{whale}
%\usecolortheme{wolverine}

%\setbeamertemplate{footline} % To remove the footer line in all slides uncomment this line
%\setbeamertemplate{footline}[page number] % To replace the footer line in all slides with a simple slide count uncomment this line

%\setbeamertemplate{navigation symbols}{} % To remove the navigation symbols from the bottom of all slides uncomment this line
}

\usepackage{graphicx} % Allows including images
\usepackage{booktabs} % Allows the use of \toprule, \midrule and \bottomrule in tables

%----------------------------------------------------------------------------------------
%	TITLE PAGE
%----------------------------------------------------------------------------------------


\title[]{Верификация асимптотической оценки временной сложности в задачах динамического программирования} % The short title appears at the bottom of every slide, the full title is only on the title page

\author{Григорянц Сергей Арменович} % Your name
\institute[] % Your institution as it will appear on the bottom of every slide, may be shorthand to save space
{Московский физико-технический институт \\
Физтех-школа Прикладной Математики и Информатики \\
Кафедра дискретной математики\\~\\ % Your institution for the title page
Научный руководитель: Дашков Евгений Владимирович
}
\date{\today} % Date, can be changed to a custom date

\begin{document}

\begin{frame}
	\titlepage
\end{frame}

\begin{frame}
	\frametitle{Постановка задачи}
	\begin{block}{Цель работы}
		\begin{itemize}
			\item<2-> Исследование методов формальной верификации корректности и асимптотики алгоритмов.
			\item<3-> Применение исследованных методов на примере верификации алгоритма динамического программирования LCS.
		\end{itemize}
	\end{block}
\end{frame}

\begin{frame}
	\frametitle{Зачем нужна формальная верификация?}
	\begin{block}{Сферы применения}
		\begin{itemize}
			\item<2-> Аппаратное обеспечение -- Intel \cite{intel}.
			\item<3-> Криптография -- Scilla \cite{Scilla}, CertiK \cite{CertiK}.
			\item<4-> Критическое ПО -- CompCert \cite{CompCert}, seL4 \cite{seL4}.
			\item<5-> Медицина, банковское дело, транспортные технологии, и т.д.
		\end{itemize}
	\end{block}
\end{frame}

\begin{frame}
	\frametitle{Что бывает, если не верифицировать ПО}
	\begin{block}{Истории неудач}
		\begin{itemize}
			\item<2-> Сорвалась миссия НАСА Mars Climate Orbiter.
			\item<3-> Ненадлежащее тестирование Лондонской службы скорой помощи привело к гибели людей.
			\item<4-> Самолет Airbus A320 разбился на демонстрационном полете из-за ошибке в софте.
			\item<5-> Много страшных историй: \cite{horror}
		\end{itemize}
	\end{block}
\end{frame}

\begin{frame}
	\frametitle{Coq}
	\begin{itemize}
		\item<2-> Coq -- программное средство доказательства теорем.
		\item<3-> Coq основан на теории типов (Исчисление Индуктивных Конструкций, Calculus of Inductive Constructions, CIC)
		\item<4-> CIC способна представлять:
		      \begin{itemize}
			      \item<5-> Функциональные программы в стиле ML.
			      \item<6-> Доказательства в логике высшего порядка.
		      \end{itemize}
		\item<7-> Утверждения и доказательства представляются с помощью Соответствия Карри — Ховарда:
		      \begin{itemize}
			      \item<8-> Пропозициональное утверждение $\iff$ Тип
			      \item<9-> Доказательство утверждения $\iff$ Элемент данного типа
		      \end{itemize}
		\item<10-> Vernacular -- язык команд Coq.
	\end{itemize}
\end{frame}

\begin{frame}
	\frametitle{Логика Хоара}
	Здесь могла быть логика Хоара. 
\end{frame}

\begin{frame}[allowframebreaks]
	\frametitle{Список литературы}
	\printbibliography
\end{frame}

%----------------------------------------------------------------------------------------

\end{document}