%%%%%%%%%%%%%%%%%%%%%%%%%%%%%%%%%%%%%%%%%
% Beamer Presentation
% LaTeX Template
% Version 1.0 (10/11/12)
%
% This template has been downloaded from:
% http://www.LaTeXTemplates.com
%
% License:
% CC BY-NC-SA 3.0 (http://creativecommons.org/licenses/by-nc-sa/3.0/)
%
%%%%%%%%%%%%%%%%%%%%%%%%%%%%%%%%%%%%%%%%%

%----------------------------------------------------------------------------------------
%	PACKAGES AND THEMES
%----------------------------------------------------------------------------------------

\documentclass{beamer}

\mode<presentation> {
\usetheme{Boadilla}

%Russian-specific packages
%--------------------------------------
\usepackage[T2A]{fontenc}
\usepackage[utf8]{inputenc}
\usepackage[russian]{babel}
\usepackage{mathrsfs}
\usepackage[backend=biber]{biblatex}
\usepackage{minted}
\usemintedstyle{tango}
\addbibresource{presentation.bib}
%--------------------------------------

% As well as themes, the Beamer class has a number of color themes
% for any slide theme. Uncomment each of these in turn to see how it
% changes the colors of your current slide theme.

%\usecolortheme{albatross}
%\usecolortheme{beaver}
%\usecolortheme{beetle}
%\usecolortheme{crane}
%\usecolortheme{dolphin}
%\usecolortheme{dove}
%\usecolortheme{fly}
%\usecolortheme{lily}
%\usecolortheme{orchid}
%\usecolortheme{rose}
%\usecolortheme{seagull}
%\usecolortheme{seahorse}
%\usecolortheme{whale}
%\usecolortheme{wolverine}

%\setbeamertemplate{footline} % To remove the footer line in all slides uncomment this line
%\setbeamertemplate{footline}[page number] % To replace the footer line in all slides with a simple slide count uncomment this line

%\setbeamertemplate{navigation symbols}{} % To remove the navigation symbols from the bottom of all slides uncomment this line
}

\usepackage{graphicx} % Allows including images
\usepackage{booktabs} % Allows the use of \toprule, \midrule and \bottomrule in tables

%----------------------------------------------------------------------------------------
%	TITLE PAGE
%----------------------------------------------------------------------------------------


\title[]{Верификация асимптотической оценки временной сложности в задачах динамического программирования} % The short title appears at the bottom of every slide, the full title is only on the title page

\author{Григорянц Сергей Арменович} % Your name
\institute[] % Your institution as it will appear on the bottom of every slide, may be shorthand to save space
{Московский физико-технический институт \\
Физтех-школа Прикладной Математики и Информатики \\
Кафедра дискретной математики\\~\\ % Your institution for the title page
Научный руководитель: Дашков Евгений Владимирович
}
\date{\today} % Date, can be changed to a custom date

\begin{document}

\begin{frame}
	\titlepage
\end{frame}

\begin{frame}
	\frametitle{Постановка задачи}
	\begin{block}{Цель работы}
		\begin{itemize}
			\item<2-> Исследование методов формальной верификации корректности и асимптотики алгоритмов, предложенных в статье Шагеро и др. \cite{base_article}.
			\item<3-> Применение исследованных методов на примере верификации алгоритма динамического программирования LCS.
		\end{itemize}
	\end{block}
\end{frame}

\begin{frame}
	\frametitle{Зачем нужна формальная верификация?}
	\begin{block}{Сферы применения}
		\begin{itemize}
			\item<2-> Аппаратное обеспечение -- Intel \cite{intel}.
			\item<3-> Криптография -- Scilla \cite{Scilla}, CertiK \cite{CertiK}.
			\item<4-> Критическое ПО -- CompCert \cite{CompCert}, seL4 \cite{seL4}.
			\item<5-> Медицина, банковское дело, транспортные технологии, и т.д.
		\end{itemize}
	\end{block}
\end{frame}

\begin{frame}
	\frametitle{Что бывает, если не верифицировать ПО}
	\begin{block}{Истории неудач}
		\begin{itemize}
			\item<2-> Сорвалась миссия НАСА Mars Climate Orbiter.
			\item<3-> Ненадлежащее тестирование Лондонской службы скорой помощи привело к гибели людей.
			\item<4-> Самолет Airbus A320 разбился на демонстрационном полете из-за ошибке в софте.
			\item<5-> Много страшных историй: \cite{horror}
		\end{itemize}
	\end{block}
\end{frame}

\begin{frame}
	\frametitle{Coq}
	\begin{itemize}
		\item<2-> Coq -- программное средство доказательства теорем.
		\item<3-> Coq основан на теории типов (Исчисление Индуктивных Конструкций, Calculus of Inductive Constructions, CIC)
		\item<4-> CIC способна представлять:
		      \begin{itemize}
			      \item<5-> Функциональные программы в стиле ML.
			      \item<6-> Логические утверждения и их формальные доказательства.
		      \end{itemize}
		\item<7-> Утверждения и доказательства представляются с помощью Соответствия Карри — Ховарда:
		      \begin{itemize}
			      \item<8-> Пропозициональное утверждение $\iff$ Тип.
			      \item<9-> Доказательство утверждения $\iff$ Элемент(терм) данного типа.
		      \end{itemize}
		\item<10-> Vernacular -- язык команд Coq.
	\end{itemize}
\end{frame}

\begin{frame}
	\frametitle{Логика Хоара}
	\begin{itemize}
		\item<2-> Логика Хоара \cite{Hoare} -- формальная системa, которая позволяет рассуждать о корректности императивных программ.
		\item<3-> Тройка Хоара $\{P\}C\{Q\}$
		      \begin{itemize}
			      \item<4-> P и Q – предикаты на множестве состояний.
			      \item<5-> C – некоторая последовательность команд.
		      \end{itemize}
		\item<6-> Тройка выводится $\iff$ если текущее состояние удовлетворяет утверждению P,
		      то после выполнения на нем программы C она будет удовлетворять утверждению Q.
		\item<7-> Пример: $\{x = 3 \land y = 4\} [x := x + y] \{x = 7 \land y = 4\}$.
		\item<8-> Логика Хоара неприменима, если в программе есть общее состояние(указатели).
		\item<9-> Пример $\{x \mapsto 3 \land y \mapsto 3\}[*x := 4]\{x \mapsto 4 \land y \mapsto ?\}$.
	\end{itemize}
\end{frame}

\begin{frame}
	\frametitle{Сепарационная логика}
	\begin{itemize}
		\item<2-> Сепарационная Логика \cite{SepLogic} -- расширение логики Хоара.
		\item<3-> Вводится дополнительна операция $P \ast Q$ -- разделяющая конъюнкция.
		      \begin{itemize}
			      \item<4-> Текущее состояние может быть разделено на два непересекающихся в адресном пространстве состояния.
			      \item<5-> Каждое из них удовлетворяет P и Q соответственно.
		      \end{itemize}
		\item<6-> Новое правило вывода(правило разделения):
		      $$
			      \frac{\{P\} C\{Q\}}{\{P * R\} C\{Q * R\}}
		      $$
		\item<7-> Пример: $\{x \mapsto 3 \ast y \mapsto 3\}[*x = 4]\{x \mapsto 4 \ast y \mapsto 3\}$.
	\end{itemize}
\end{frame}

\begin{frame}
	\frametitle{Сепарационная Логика с временными ресурсами}
	\begin{itemize}
		\item<2-> Основная идея \cite{SepLogicTime} -- введение времени как потребляемого ресурса внутрь предиката состояния.
		\item<3-> Предикат $\$1$ означает возможность сделать один шаг вычисления.
		\item<4-> Предикат $\$n$ -- разделяющая конъюнкция n предикатов $\$1$.
		\item<5-> Ресурс поглощается при выполнении шага вычисления.
		\item<6-> Пример: $\{\$10 \ast x \mapsto 3 \ast y \mapsto 3\}[*x = 4]\{\$9 \ast x \mapsto 4 \ast y \mapsto 3\}$.
	\end{itemize}
\end{frame}

\begin{frame}[fragile]
	\frametitle{Реализация LCS}
	\begin{minted}{ocaml}
let lcs (a : int array) (b : int array) : int array =
  let n = Array.length a in
  let m = Array.length b in
  let c = Array.make ((n+1)*(m+1)) [] in
  for i = 1 to n do
    for j = 1 to m do
      if a.(i-1) = b.(j-1)
      then c.(i*(m+1) + j) <- 
      	List.append c.((i-1)*(m+1) + j - 1) [a.(i-1)]
      else if List.length c.((i-1)*(m+1) + j) > 
      	List.length c.(i*(m+1) + j - 1)
        then c.(i*(m+1) + j) <- c.((i-1)*(m+1) + j)
        else c.(i*(m+1) + j) <- c.(i*(m+1) + j - 1)
      done
    done; 
  Array.of_list c.((n+1)*(m+1)-1);;
	\end{minted}
\end{frame}

\begin{frame}[fragile]
	\frametitle{Определение SubSeq}
	\begin{minted}{coq}
Inductive SubSeq {A:Type} : list A -> list A -> Prop :=
 | SubNil (l:list A) : SubSeq nil l
 | SubCons1 (x:A) (l1 l2:list A) (H: SubSeq l1 l2) : 
 		SubSeq l1 (x::l2)
 | SubCons2 (x:A) (l1 l2:list A) (H: SubSeq l1 l2) : 
 		SubSeq (x::l1) (x::l2).
	\end{minted}
\end{frame}

\begin{frame}[fragile]
	\frametitle{Определение Lcs}
	\begin{minted}{coq}
Definition Lcs {A: Type} l l1 l2 :=
  SubSeq l l1 /\ SubSeq l l2 /\ 
  (forall l': list A, SubSeq l' l1 /\ SubSeq l' l2 -> 
  	length l' <= length l). 
	\end{minted}
\end{frame}

\begin{frame}[fragile]
	\frametitle{Необходимые свойства Lcs}
	\begin{minted}{coq}
Lemma lcs_app_eq: forall (l1 l2 l: list int) (x: int),
  Lcs l l1 l2 -> Lcs (l & x) (l1 & x) (l2 & x). 

Lemma lcs_app_neq: forall (l1 l2 l l': list int) (x y : int),
  x <> y -> Lcs l (l1&x) l2 -> Lcs l' l1 (l2&y) -> 
  	length l' <= length l -> Lcs l (l1&x) (l2&y). 
	\end{minted}
\end{frame}

\begin{frame}[fragile]
	\frametitle{Формулировка основной теоремы}
	\begin{minted}{coq}
Lemma lcs_spec:
  specO
    (product_filterType Z_filterType Z_filterType)
    ZZle
  ( fun cost =>
  forall (l1 l2 : list int) p1 p2,
  app lcs [p1 p2]
  PRE (\$(cost (LibListZ.length l1, LibListZ.length l2)) \*
  p1 ~> Array l1 \* p2 ~> Array l2)
  POST (fun p => Hexists (l : list int), p ~> Array l \* 
  	\[Lcs l l1 l2]))
  (fun '(n,m) => n * m).
	\end{minted}
\end{frame}

\begin{frame}[fragile]
	\frametitle{Инвариант цикла}
	\begin{minted}[]{coq}
xfor_inv (fun (i:int) => 
Hexists (x' : list (list int)),
p1 ~> Array l1 \*
p2 ~> Array l2 \*
c ~> Array x' \*
\[length x' = (n+1)*(m+1)] \*
\[forall i1 i2 : int, 0 <= i1 < i -> 0 <= i2 <= m -> 
Lcs x'[i1*(m+1) + i2] (take i1 l1 ) (take i2 l2) ] \* 
\[forall i', i*(m+1) <= i' < (n+1)*(m+1) ->
x'[i'] = nil ]). 
	\end{minted}
\end{frame}

\begin{frame}
	\frametitle{Возможные темы дальнейшего исследования}
	\begin{itemize}
		\item<2-> Реализация функции \textcolor{blue}{HForall}.
		      \begin{itemize}
			      \item<3-> Реализовать саму функцию.
			      \item<4-> Доказать ее свойства для интеграции с тактикой \textcolor{blue}{hsimpl}.
		      \end{itemize}
		\item<5-> Формальная верификация асимптотики вероятностных алгоритмов.
	\end{itemize}
\end{frame}

\begin{frame}[allowframebreaks]
	\frametitle{Список литературы}
	\printbibliography
\end{frame}

%----------------------------------------------------------------------------------------

\end{document}